\documentclass[11pt,english]{article}
\usepackage[T1]{fontenc}
\usepackage[latin9]{inputenc}
\usepackage{babel}
\usepackage{verbatim}
\usepackage{float}
\usepackage{amsmath}
\usepackage{amssymb}
\usepackage{setspace}
\usepackage{graphicx}
\usepackage{subfigure}
\usepackage{pdfpages}

\PassOptionsToPackage{normalem}{ulem}
\usepackage{ulem}
\onehalfspacing
%\usepackage[unicode=true,pdfusetitle,
% bookmarks=true,bookmarksnumbered=false,bookmarksopen=false,
% breaklinks=false,pdfborder={0 0 1},backref=section,colorlinks=false]
% {hyperref}
\usepackage{hyperref}
%\usepackage{breakurl}


%%%%%%%%%%%%%%%%%%%%%%%%%%%%%% User specified LaTeX commands.

\usepackage{latexsym}
\addtolength{\hoffset}{-0.75in}
\addtolength{\voffset}{-0.75in}
\addtolength{\textwidth}{1.5in}
\addtolength{\textheight}{1.6in}

%\usepackage{Sweave}
%\include{Sweave}
%\usepackage{listings}

% === dcolumn package ===
\usepackage{dcolumn}\newcolumntype{.}{D{.}{.}{-1}}
\newcolumntype{d}[1]{D{.}{.}{#1}}

\usepackage{comment}

% === more new math commands
\newcommand{\X}{\mathbf{X}}
\newcommand{\Y}{\mathbf{Y}}
\renewcommand{\r}{\right}
\renewcommand{\l}{\left}
\newcommand{\dist}{\buildrel\rm d\over\sim}
\newcommand{\ind}{\stackrel{\rm indep.}{\sim}}
\newcommand{\ud}{\mathrm{d}}
\newcommand{\iid}{\stackrel{\rm i.i.d.}{\sim}}
\newcommand{\logit}{{\rm logit}}
\newcommand{\cA}{\mathcal{A}}
\newcommand{\E}{\mathbb{E}}
\newcommand{\V}{\mathbb{V}}
\newcommand{\cJ}{\mathcal{J}}
\newcommand{\bone}{\mathbf{1}}
\newcommand{\var}{{\rm Var}}
\newcommand{\cov}{{\rm Cov}}
\newcommand{\tomega}{\tilde\omega}

% === spacing
\newcommand{\spacingset}[1]{\renewcommand{\baselinestretch}%
{#1}\small\normalsize}
% \spacingset{1.2}
\setlength{\parindent}{0pt}
\setlength{\parskip}{2em}
%%%%%%%%%%%%%%%%%%%%%%%%%%%%%%%%%%%%%%%%%%%%%%%%%%%%%%%%%%%%%%%%%%%%%%%%%%%%%%%

\begin{document}

1)Let Z be binomial random variable so that if $Z=1$, deaths follows Poisson($\mu_1$) and if $Z=2$, deaths follows Poisson($\mu_2$).  Obviously, Z is the latent variable which we cannot observe.  First we calculate the conditional distribution $(Z|X,\theta)$:
\begin{flalign*}
&P(Z_i=1 | X=x_i,\theta_m) = \frac{\alpha_m \frac{\mu_{m,1}e^{-\mu_{m,1}}}{x_i!} }{ \alpha_m \frac{\mu_{m,1}e^{-\mu_{m,1}}}{x_i!} + (1-\alpha_m) \frac{\mu_{m,2}e^{-\mu_{m,2}}}{x_i!}} = z_{x_i}(\theta_m)& \\
&P(Z_i=2 | X=x_i,\theta_m) = \frac{(1-\alpha_m) \frac{\mu_{m,2}e^{-\mu_{m,2}}}{x_i!} }{ \alpha_m \frac{\mu_{m,1}e^{-\mu_{m,1}}}{x_i!} + (1-\alpha_m) \frac{\mu_{m,2}e^{-\mu_{m,2}}}{x_i!}} = 1-  z_{x_i}(\theta_m)& \\
\end{flalign*}
Now, we calculate the $Q(\theta|\theta_m)$.
\begin{flalign*}
&Q(\theta|\theta_m) = \E_{Z|X,\theta_m} \left[ \log L(\theta;X,Z)\right] = \sum_i^n \E_{Z|X,\theta_m} \left[ \log L(\theta;x_i,z_i)\right]&\\
&= \sum_i^n\left[P(Z_i=1 | X=x_i,\theta_m)\log L(\theta;x_i,z_i) +  P(Z_i=2 | X=x_i,\theta_m)\log L(\theta;x_i,z_i)\right]&\\
&= \sum_i^n \left[  z_{x_i}(\theta_m)* \log(\alpha \frac{\mu_1^{x_i}e^{-\mu_1}}{x_i!}) + (1- z_{x_i}(\theta_m))*  \log((1-\alpha) \frac{\mu_2^{x_i}e^{-\mu_2}}{x_i!}) \right]&\\
&=\sum_i^n \left[z_{x_i}(\theta_m)*(\log\alpha + x_i\log\mu_1-\mu_1 - \log x_i!) + (1-z_{x_i}(\theta_m))*(\log(1-\alpha) + x_i\log\mu_2-\mu_2 - \log x_i!) \right]&
\end{flalign*}
Calculate the maximum point $\theta_{m+1}$.
\begin{flalign*}
&\frac{\partial Q(\theta|\theta_m)}{\partial \mu_1} = \sum_i^n z_{x_i}(\theta_m)\left[ \frac{x_i}{\mu_1} -1 \right] = \sum_{i=0}^9 n_iz_i(\theta_m)\left[ \frac{i}{\mu_1} -1\right]= 0 &\\
&\Rightarrow \mu_{m+1,1} =  \frac{\sum_{i=0}^9 z_i(\theta_m)i \theta_m}{\sum_{i=0}^9 z_i(\theta_m) \theta_m}&\\
&\frac{\partial Q(\theta|\theta_m)}{\partial \mu_2} = \sum_i^n (1-z_{x_i}(\theta_m))\left[ \frac{x_i}{\mu_2} -1 \right] = \sum_{i=0}^9 n_i(1-z_i(\theta_m))\left[ \frac{i}{\mu_2} -1\right]= 0&\\
&\Rightarrow \mu_{m+1,2} =  \frac{\sum_{i=0}^9 (1-z_i(\theta_m))i \theta_m}{\sum_{i=0}^9 (1-z_i(\theta_m)) \theta_m}&\\
&\frac{\partial Q(\theta|\theta_m)}{\partial \alpha} = \sum_i^n \left[ \frac{z_{x_i}(\theta_m)}{\alpha} - \frac{1 -z_{x_i}(\theta_m) }{1-\alpha}\right] = \sum_{i=0}^9 n_i\left[ \frac{z_{i}(\theta_m)}{\alpha} - \frac{1 -z_{i}(\theta_m)}{1-\alpha}\right]=0&\\
&\Rightarrow \alpha_{m+1}=\frac{\sum_{i=0}^9n_iz_{i}(\theta_m) }{\sum_{i=0}^9{n_i}} &
\end{flalign*}
Proof is finished.\\










\end{document}

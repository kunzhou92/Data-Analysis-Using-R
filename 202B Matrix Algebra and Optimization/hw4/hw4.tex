\documentclass[11pt,english]{article}
\usepackage[T1]{fontenc}
\usepackage[latin9]{inputenc}
\usepackage{babel}
\usepackage{verbatim}
\usepackage{float}
\usepackage{amsmath}
\usepackage{amssymb}
\usepackage{setspace}
\usepackage{graphicx}
\usepackage{subfigure}
\usepackage{pdfpages}

\PassOptionsToPackage{normalem}{ulem}
\usepackage{ulem}
\onehalfspacing
%\usepackage[unicode=true,pdfusetitle,
% bookmarks=true,bookmarksnumbered=false,bookmarksopen=false,
% breaklinks=false,pdfborder={0 0 1},backref=section,colorlinks=false]
% {hyperref}
\usepackage{hyperref}
%\usepackage{breakurl}


%%%%%%%%%%%%%%%%%%%%%%%%%%%%%% User specified LaTeX commands.

\usepackage{latexsym}
\addtolength{\hoffset}{-0.75in}
\addtolength{\voffset}{-0.75in}
\addtolength{\textwidth}{1.5in}
\addtolength{\textheight}{1.6in}

%\usepackage{Sweave}
%\include{Sweave}
%\usepackage{listings}

% === dcolumn package ===
\usepackage{dcolumn}\newcolumntype{.}{D{.}{.}{-1}}
\newcolumntype{d}[1]{D{.}{.}{#1}}

\usepackage{comment}

% === more new math commands
\newcommand{\X}{\mathbf{X}}
\newcommand{\Y}{\mathbf{Y}}
\renewcommand{\r}{\right}
\renewcommand{\l}{\left}
\newcommand{\dist}{\buildrel\rm d\over\sim}
\newcommand{\ind}{\stackrel{\rm indep.}{\sim}}
\newcommand{\ud}{\mathrm{d}}
\newcommand{\iid}{\stackrel{\rm i.i.d.}{\sim}}
\newcommand{\logit}{{\rm logit}}
\newcommand{\cA}{\mathcal{A}}
\newcommand{\E}{\mathbb{E}}
\newcommand{\V}{\mathbb{V}}
\newcommand{\cJ}{\mathcal{J}}
\newcommand{\bone}{\mathbf{1}}
\newcommand{\var}{{\rm Var}}
\newcommand{\cov}{{\rm Cov}}
\newcommand{\tomega}{\tilde\omega}

% === spacing
\newcommand{\spacingset}[1]{\renewcommand{\baselinestretch}%
{#1}\small\normalsize}
% \spacingset{1.2}
\setlength{\parindent}{0pt}
\setlength{\parskip}{2em}
%%%%%%%%%%%%%%%%%%%%%%%%%%%%%%%%%%%%%%%%%%%%%%%%%%%%%%%%%%%%%%%%%%%%%%%%%%%%%%%

\begin{document}

2)
\begin{flalign*}
&tr\left[ D_r^{-\frac{1}{2}}(P - \widehat{P})D_c^{-1}(P - \widehat{P})^t D_r^{-\frac{1}{2}}\right] = tr\left[ \left( D_r^{-\frac{1}{2}}(P-\widehat{P})D_c^{-\frac{1}{2}}\right)\left( D_r^{-\frac{1}{2}}(P-\widehat{P})D_c^{-\frac{1}{2}}\right)^t\right]&\\
\end{flalign*}
By SVD, $D_r^{-\frac{1}{2}}PD_c^{-\frac{1}{2}} = U\Lambda V^t$.\\
By Eckart-Young Theorem, the best t-rank reduced matrix $D_r^{-\frac{1}{2}}\widehat{P}D_c^{-\frac{1}{2}} =  U\Lambda_t V^t$, where the first t diagonal elements of $\Lambda_t $ are same with those of $\Lambda$ and other elements are zeros.  Thus, $\widehat{P} = D_r^{\frac{1}{2}}U\Lambda_t V^tD_c^{\frac{1}{2}}$.\\
When $t=1$, $D_r^{-\frac{1}{2}}PD_c^{-\frac{1}{2}} = \left(\frac{n_{ij}}{\sqrt{n_i}\sqrt{n_j}}\right)_{ij}$. Meanwhile we have $D^{-\frac{1}{2}}_r r = (\sqrt{\frac{n_{i+}}{n}})_i$ and $D^{-\frac{1}{2}}_c c = (\sqrt{\frac{n_{+j}}{n}})_j$.
\begin{flalign*}
&\left( D_r^{-\frac{1}{2}}PD_c^{-\frac{1}{2}} \right)\left( D_r^{-\frac{1}{2}}PD_c^{-\frac{1}{2}} \right)^t \left( D^{-\frac{1}{2}}_r r\right) = \left(\sum_{k,j}\frac{n_{ik}}{\sqrt{n_{i+}n_{+k}}} \frac{n_{jk}}{\sqrt{n_{j+}n_{+k}}}\sqrt{\frac{n_{j+}}{n}} \right)_i =\left(\sqrt{\frac{n_{i+}}{n}}\right)_i = D^{-\frac{1}{2}}_r r&\\
&\left( D_r^{-\frac{1}{2}}PD_c^{-\frac{1}{2}} \right)^t\left( D_r^{-\frac{1}{2}}PD_c^{-\frac{1}{2}} \right) \left( D^{-\frac{1}{2}}_c c\right) = \left(\sum_{k,i}\frac{n_{kj}}{\sqrt{n_{k+}n_{+j}}} \frac{n_{ki}}{\sqrt{n_{k+}n_{+i}}}\sqrt{\frac{n_{+i}}{n}} \right)_j =\left(\sqrt{\frac{n_{+j}}{n}}\right)_j = D^{-\frac{1}{2}}_c c&
\end{flalign*}
So we know $D_r^{-\frac{1}{2}}PD_c^{-\frac{1}{2}}$ have singular value $\lambda = 1$.  We denote $D_r^{-\frac{1}{2}}PD_c^{-\frac{1}{2}}$ as $A$ and we know its element $\frac{n_{ij}}{\sqrt{n_{i+}n_{+j}}} \leq 1$.  So $\lim_n(AA^t)^n = B$ and each row and column of B is either 1 or 0.  If A had a singular value greater than 1, this would not happen.  So $\lambda = 1$ is the greatest singular value. Namely, 
$$D_r^{-\frac{1}{2}}PD_c^{-\frac{1}{2}} = (D^{-\frac{1}{2}}_r r | U_{n*(r-1)}) \begin{pmatrix}
1& 0\\
0 & \Lambda'
\end{pmatrix} (D_c^{-\frac{1}{2}}c|V_{n*(s-1)})^t$$
When rank $t=1$, $D_r^{-\frac{1}{2}}\widehat{P}D_c^{-\frac{1}{2}} = D^{-\frac{1}{2}}_r r c^t D_c^{-\frac{1}{2}} \Rightarrow \widehat{P} = rc^t$.












\end{document}
